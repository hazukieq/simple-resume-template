%请前往 通用配置.tex 文件内进行修改!
\addBg % 添加背景色命令

% 简历开头标题,使用作者名
\rTitle{\theAuthorName}

%=======基本信息版块=======%%
%使用表格呈现个人信息
%这里请前往 通用配置.tex 文件内进行修改!
\begin{table}[h]
	\begin{tabular}{p{1in}p{12em}p{\the\dimexpr\linewidth-1.3in-12em\relax}}
		% 创建一个三列的表格
		% 第一列宽1英寸,第二列宽12em,第三列宽为剩余宽度
		\includegraphics[width=1in]{\theAvatarPath} & % 插入头像,路径内容由变量 \theAvatarPath 定义
		\parbox[b]{12em}{ % 创建一个宽12em 的文本框,用于显示个人基本信息
			\zihao{-4} % 设置小四号字体
			\textbf{年\quad 龄:}\theAge 岁 \\ % 年龄信息
			\textbf{性\quad 别:}\theGender \\ % 性别信息
			\textbf{身\quad 高:}\theHeight \\ % 身高信息
			\textbf{籍\quad 贯:}\theHometown % 籍贯信息
			\vspace{1em} % 在该文本框底部添加1em 的垂直间距
		} & 
		\parbox[b]{12em}{ % 创建第二个宽12em 的文本框,用于显示更多的个人基本信息
			\zihao{-4} % 继续使用小四号字体
			\textbf{政治面貌:}\thePolitics \\ % 政治面貌信息
			\textbf{学\qquad 历:}\theStudyMode \\ % 学历信息
			\textbf{邮\qquad 箱:}\theEmail \\ % 邮箱信息
			\textbf{电\qquad 话:}\thePhoneNumber % 电话信息
			\vspace{1em} % 在该文本框底部添加1em 的垂直间距
		} \\
	\end{tabular}
\end{table} % 结束表格环境
%=======基本信息版块=======%%


% 与上一个版块的间距调整,降低间距
\vspace{-1em} 
\zihao{-4} % 设置后续内容为小四号字体

%=======教育经历版块=======%%
\spect{教育经历} 
{\textbf{\theUniversityName}}\\[3pt] % 加粗院校名称
\litem{学士学位,主修\theMajor}{{\kaishu 2020.9-2025.7}}\\[0.2em] % 列出学位、专业和学习时间
% 主修课程说明
主修课程:\fsong{角色设计、场景构建、游戏编程、虚拟现实、互动媒介、游戏心理学、3D建模、动画制作等} \\
学分绩点:3.75 \qquad 排名:前 20\%(12/60) % 显示绩点和排名
\vspace*{6pt} % 添加6pt的垂直间距

% 列出相关经历
\litem{游戏开发小组\quad \textbf{组长}}{2022.6-2023.6} % 列出担任的职位和期间
\descript{ % 描述这项经历
	\indent 监督和组织小组项目,带领组员完成多个游戏设计和开发任务 % 具体责任介绍
}

% 另一个相关经历
\litem{原神玩家社团\quad \textbf{社长}}{2022.6-2023.6} % 列出另一个担任的职位和期间
\descript{ % 描述这项经历
	\indent 组织社团活动,策划游戏比赛和讨论会,提高社团成员的游戏体验和合作能力 % 具体责任介绍
}
%=======教育经历版块=======%%



%=======实习经历版块=======%%
\spect{实习经历} 
\litem{原神游戏开发公司,游戏实习生}{2023.7-2024.4} % 列出实习单位、职位和时间
\descript{ % 描述实习内容
	\indent 参与角色设计和场景搭建,实现游戏内的各类机制,提升用户交互体验,熟悉使用Unity和Unreal Engine等开发工具 % 具体责任与收获
}
%=======实习经历版块=======%%


%=======项目经历版块=======%%
\spect{项目经历} 
\litem{“原神角色造型设计”项目,负责人}{2023.11} % 列出项目名称和担任角色
\begin{pblock} % 使用块环境描述项目成果
	\item {
		\zihao{5} % 设置该项目描述字体为五号
		开发并设计了多个全新角色造型,参与角色背景故事和技能设计,项目获评优秀。 % 项目的具体工作和评价
	}
\end{pblock}

\vspace{4pt} % 添加4pt的垂直间距
\litem{“虚拟风景体验”项目,成员}{2023.7} % 列出另一个项目
\begin{pblock} % 项目描述
	\item 参与开发基于游戏引擎的虚拟风景展示项目,负责场景细节优化和交互设计。 % 项目的具体工作
\end{pblock}

\vspace{4pt} % 添加4pt的垂直间距
\litem{“跨界互动艺术”项目,成员}{2024.3} % 列出另一个项目
\begin{pblock} % 项目描述
	\item 参与原神主题的跨界互动艺术展览,负责视觉设计和互动体验的实现。 % 项目的具体工作
\end{pblock}
%=======项目经历版块=======%%


%=======技能与证书版块=======%%
\spect{技能与证书} 
\begin{plist} % 列表环境开始
	\item \textbf{技能:}熟练使用Unity和Unreal Engine,掌握Blender和Maya 3D建模,了解游戏机制设计和用户体验 % 列出技能
	\item \textbf{证书:}计算机高级证书、游戏开发专业证书 % 列出获得的证书
	\item \textbf{奖状:} % 奖状列表开始
	\begin{pitem}
		\item 原神游戏设计大赛二等奖\quad\qquad\qquad 实习生创新项目评选优秀 % 列举的获奖情况
		\item 学校“最佳团队合作奖”\qquad\qquad\qquad\quad “数字艺术创新奖” 
	\end{pitem}
\end{plist} % 列表环境结束
%=======技能与证书版块=======%%



%=======个人总结版块=======%%
\spect{个人总结} 
\begin{pitem} % 项目环境开始
	\item 具备良好的学习能力和适应性,能在快速变化的环境中工作 % 第一点总结
	\item 工作认真负责,具备优秀的团队合作意识和沟通能力 % 第二点总结
	\item 热爱游戏制作,积极参与相关活动,平时喜欢玩游戏和分享游戏设计理念 % 第三点总结
\end{pitem} % 项目环境结束
%=======个人总结版块=======%%