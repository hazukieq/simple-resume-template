% !TeX document-id = {867351ff-cf8c-4583-934e-ee0bcf1335f9}
%!TeX TXS-program:compile = txs:///xelatex/[-8bit -shell-escape]
\documentclass[a4paper,12pt]{ctexart}
\usepackage{xeCJK}
\usepackage{graphicx}
\usepackage{enumitem}
\usepackage{tabularx}
\ctexset{
	today=small,
	section/name={第,节},
	section/number=\arabic{section},
	section/format+=\raggedright,
	subsection/aftername=\hskip 0.5em,
	subsubsection/aftername=\hskip 0.5em,
}

%设置页面边距
\usepackage[
top=2.5cm,bottom=2.5cm,
left=2cm,right=2cm,
]{geometry}

\setlength\headheight{52pt}
%段落间距
\linespread{1.3}


\title{简易简历模板}
\author{叶月绘梨依}



\usepackage{fancyhdr}
\pagestyle{fancy}
%\renewcommand{\chaptermark}[1]{\markboth{#1}{}}
\renewcommand{\sectionmark}[1]{\markright{\CTEXifname{\CTEXthesection}{}\ #1}}
\renewcommand{\subsectionmark}[1]{} % 不为 subsection 设置标记


\fancyhf{}
\fancyhead[L]{\textnormal{\kaishu \rightmark}}
\fancyhead[R]{\textnormal{\kaishu \thepage}}
\renewcommand{\headrulewidth}{.5pt} %注意不用 \setlength
\renewcommand{\footrulewidth}{0pt}

\usepackage[
bookmarks=true,
colorlinks=true,
pdfstartview=fitH
]{hyperref}

\newenvironment{pol}[1]{
	\begin{enumerate}[topsep=0pt,labelsep=.5em,leftmargin=\the\dimexpr 0.5em+ #1\relax,itemsep=0em,parsep=0em, partopsep=0pt,label=\arabic*.]
		
	}{\end{enumerate}}


\usepackage{enumitem}
\newenvironment{pul}[1]{
	\begin{itemize}[topsep=0pt,labelsep=.5em,leftmargin=\the\dimexpr 0.5em+ #1\relax,itemsep=0em,parsep=.5em]
	}{\end{itemize}}

\newenvironment{ul}{
	\begin{itemize}[topsep=0pt,labelsep=.5em,leftmargin=4.5em,itemsep=-.5em]
	}{\end{itemize}}

\newenvironment{numul}{
	\begin{enumerate}[left=0em]
	}{\end{enumerate}}

% 设置图像文件夹
\graphicspath{{./statics/}}

\newcommand{\pic}[1]{
	\begin{figure}[!h]
		\centering
		\includegraphics[width=4in]{#1}
	\end{figure}
}

\newcommand{\fpic}[1]{
	\begin{figure}[!h]
		\centering
		\fbox{\includegraphics[width=4in]{#1}}
	\end{figure}
}

\usepackage{xeCJKfntef} %下划线自动换行
\usepackage{algorithm}
\usepackage{minted}
\setminted{tabsize=4}
\usepackage{listings}% 引入listings包,用于在文档中插入代码,并可自定义代码样式
\usepackage{xcolor}
\usepackage{tcolorbox}


\definecolor{codeBg}{HTML}{f6f8fa}
\definecolor{emphColor}{HTML}{ef259d}
% 定义自定义代码环境
\newminted{bash}{bgcolor=codeBg,frame=leftline,framesep=0em,framerule=.1em,rulecolor=pink,fontsize=\normalsize}
%linenos,numbers=right,numbersep=-1em
\newminted{latex}{bgcolor=codeBg,frame=leftline,framesep=0em,framerule=.1em,rulecolor=pink,fontsize=\normalsize}

\definecolor{spanbg}{HTML}{eeeeee}

\newcommand{\cbox}[1]{
	\tcbox[colback=spanbg, colframe=spanbg, rounded corners=all]{#1}
}

\newcommand{\emp}[1]{
	\colorbox{spanbg}{#1}
}

\begin{document}
	\maketitle
	\setcounter{page}{0}
	\thispagestyle{empty}
	
	\newpage
	\setcounter{page}{1}
	\pagenumbering{arabic}
	\section{快速入门}
	\subsection{项目简介}
	模板在制作过程中,借鉴了以下 Github 项目:
	\begin{ul}
		\item\href{https://github.com/dyweb/awesome-resume-for-chinese}{awesome-resume, @dyweb}
		\item\href{https://github.com/dyweb/Deedy-Resume-for-Chinese}{Deedy-Resume-for-Chinese, @dyweb}
		\item\href{https://github.com/geekplux/cv_resume}{cv-resume, @geekplux}
		\item\href{https://github.com/billryan/resume}{resume, @billryan}
		\item\href{https://github.com/hijiangtao/resume}{resume, @hijiantao}
	\end{ul}
	
	在这些简历的基础上,编写了符合个人风格的简历模板。模板布局主要参考自 \textbf{billryan}的排版样式,版块分布则参照了学院简历书写规范及上述项目简历内容。通过查阅\LaTeX{}文档及互联网问答论坛(如知乎、StackExchange等),经过3天的时间,编辑出了令自己较为满意的\LaTeX 模板。
	

	该模板致力于:
	\begin{ul}
		\item 提供基本信息控制,通过修改基本信息实现快速魔改
		\item 关注内容,而非格式。时间应花费在内容编排
		\item 提供 Makefile 编译,支持生成附件(奖状证件)的信息索引,方便内容迁移
		%\item 保证简历组成清晰合理,利于管理维护
	\end{ul}
	
	\subsection{项目缘起}
	由于大四实习前制作的简历,太过简陋,加之实习那年11月签了某单位后,就将找工作的简历丢之角落尘封许久。实习结束后,因家中变故及个人对于医院的反感情绪影响,做出了“短浅”的决定——即解除相关合同。于是,自己不得不找工作,重拾了简历后才始知当年制作水准有多么一言难尽。将就着向其他行业投了几十份简历,杳无音信后,察觉简历内容和形式或是影响找工作的因素之一,自己就有了重制的打算。加之,自己在五六月份赶论文期间积累了不少排版心得。趁着赋闲在家,边学习\LaTeX 基本语法,边着手简历重制工作。
	
	\subsection{开始使用}
	\textbf{前期准备}
	
	\qquad 由于模板基于\LaTeX 编写,所以要求计算机上安装有 TeX且配置相关环境变量。注意:若想使用附件索引生成功能,\textcolor{red}{请务必安装有支持 Posix 规范的终端模拟器}。目前相关脚本暂无支持windows的批处理脚本。
	
	\qquad 以下是项目完全运行的最小依赖(推荐):`
	\begin{pul}{6em}
		\item TexLive2025
		\item Make(4.3)
		\item Bash
	\end{pul}

	\pagebreak
	
	\qquad \textcolor{purple}{Windows 安装建议}
	\begin{pul}{6em}
		\item
		TexLive: 下载 \href{https://mirrors.tuna.tsinghua.edu.cn/CTAN/systems/texlive/Images/texlive2025-20250308.iso}{TexLive ISO光盘},双击文件,找到install-tl-windows.bat,按照提示安装,在cmd上输入 \mintinline{bash}{tex -v} 后输出版本信息则证明安装成功
		\item \href{https://www.cygwin.com/setup-x86_64.exe}{Cygwin}: 一款支持 Posix规范的终端,安装时可选 make。安装教程可参考\href{https://zhuanlan.zhihu.com/p/474242350}{这篇文章}。
	\end{pul}
	
	\qquad \textcolor{purple}{Ubuntu 安装建议}
	\begin{pul}{6em}
		\item
		TexLive: 下载 \href{https://mirrors.tuna.tsinghua.edu.cn/CTAN/systems/texlive/Images/texlive2025-20250308.iso}{TexLive ISO光盘}
		
		为方便安装,推荐在终端环境下操作。
		
		假设 ISO光盘路径:\textbf{/home/ubuntu/texlive2025.iso}
		
		将其挂载到 /mnt 目录:
		
		\begin{bashcode}
    sudo mount -o loop /home/ubuntu/texlive2025.iso /mnt
		\end{bashcode}
		
		然后跳转到 /mnt 目录,并运行安装脚本
		\begin{bashcode}
	cd /mnt
	sudo ./install-tl
		\end{bashcode}
		请仔细查阅提示,输入 \textbf{I} 后回车安装。
		\pic{texlive-install-hint.png}
		
		安装后输出信息包含有环境变量,请暂时记住,后面配置要用到。
		\pic{texlive-envs.png}
		
		移除 ISO光盘挂载:
		\begin{bashcode}
	sudo umount /mnt
		\end{bashcode}
		
		并将环境变量(如上图)添加到 .bashrc或.zshrc 中。环境变量添加操作请自行检索。
		
		至此 TexLive 安装完成。
		
		
		\item
		安装 make 库
		\begin{bashcode}
	sudo apt install make
	make -v #检验是否安装成功:输出版本信息
		\end{bashcode} 
	
	\end{pul}
	 
	\textbf{下载项目代码}
	\begin{ul}
		\item 可以通过点击项目右侧 \textbf{Release},找到压缩包后点击下载
		\item
		通过终端 git 远程拉取代码:
		\begin{bashcode}[xleftmargin=2em]
  git clone https://github.com/hazukieq/simple-resume-template.git	#请打开 bash/powershell等,输入此条指令
		\end{bashcode}
	\end{ul}
	
	\textbf{编译模板}

	\qquad 手动下载者解压后,需在当前文件夹内打开终端
	\begin{bashcode}[xleftmargin=4em]
  make #编译输出PDF文件
	\end{bashcode}
	
	\textbf{查看 PDF 文件}
	
	\qquad 编译完成后,当前文件夹下会产生一个 PDF 文件。你可以使用阅读器进行查看——Ubuntu可使用系统内置的Evince,Windows可用Edge打开。
		
	\subsection{Makefile 编译选项}
	\begin{bashcode}
  使用说明:
	make              编译 PDF 文件
	make all          等同于 'make'
	make collect      插入汇总奖状附件信息,按新到旧顺序排列
	make rcollect     插入汇总奖状附件信息,按旧到新顺序排列
	make ncollect     按附件文中出现顺序排列
	make restore      删除自动插入的内容(可手动在 tex 文件中删除)
	make clean        清理编译生成的临时文件
	make help         显示此帮助信息
	
	\end{bashcode}
	
	\subsection{许可证 License}
	Apache 2.0 协议开源,图片资源、字体部分详见其对应开源协议。\textbf{仅用于教育学习目的,不得作为商业用途。}
	
	\newpage
	\section{页面排布说明}
	
	\subsection{简历构成}
	简历模板由封面页、自荐信页、简历页、附件索引页、附件页组成。
	其中,封面页、自荐信页、附件索引页、附件页是否添加可以在\emp{通用配置.tex}中更改。
	
	\textcolor{red}{注意:配置选项中 true/false 表示需要/不需要,请不要输入其他文字,否则会导致编译失败!}
	在通用配置.tex中找到这些选项,设置是否开启:
	\begin{latexcode}
    \def\theCoverPageNeed{false} %是否需要封面页
    \def\theCoverLetterNeed{true}  %是否需要自荐信
    \def\theAttachmentIndexPageNeed{true} %是否需要附件索引页
    \def\theAttachmentPageNeed{true}  %是否需要附加页
	\end{latexcode}
	
	\textbf{定制个性化简历},只需要打开相应文件,对内容编辑,实时编译后即可查看最新更改。如:想更改自荐信,则打开 \emp{自荐信.tex},修改内容并保存,在\emp{TeXStudio}等编辑器或\emp{Gnome Terminal}等终端上点按编译。下面对简历组成页面进行一一说明。
	
	\subsection{具体设置}
	\subsubsection{封面页}
	模板封面偏简约风,由\emp{院校名称+应届生}标题和\emp{作者姓名}、\emp{日期}三部分构成。由于院校名称、作者姓名、日期属于配置信息,原则上需要前往\emp{通用配置.tex}进行设置,不建议在封面内修改(容易扰乱布局排版)。
	\begin{latexcode}
	\def\theUniversityName{大学名}  % 所在大学名称
	\def\theCollegeName{学院名}  % 所在学院名称
	\def\theSession{xxxx届}  % 届数
	\def\theAuthorName{你的姓名}  % 作者姓名
	\def\theReleaseDate{发布日期} %可选项有[具体日期|自动获取:\today]	
	\end{latexcode}
	
	当然,若不满意或拥有自己设计的封面,也可以进行替换,替换文件格式不能是\emp{doc、docx}。为保持清晰度一致,推荐顺序为:PDF>JPEG>PNG。下面是设置自定义封面的步骤:
	\begin{pul}{2em}
	\item 准备好符合格式的封面文件(假设为\emp{hazukie-cover.pdf},请自行替换为\emp{实际文件名})
	\item  将\emp{封面文件}拖入当前文件夹的\emp{imgs}文件夹中
	\item 打开\emp{通用配置.tex},找到选项:
		\begin{latexcode}
	%设置封面时,请确保这个选项是开启的!
	\def\theCoverPageNeed{true} %是否需要封面页
	%是否需要自定义封面
	\def\theCoverPageCustomNeed{true}  
	%自定义封面的文件路径
	\def\theCoverPageCustomPath{imgs/hazukie-cover.pdf}  
	\def\theCoverPageCustomScale{1} %自定义封面的缩放系数:0-1
	
		\end{latexcode}
	\item 点按编译
	\end{pul}
	\subsubsection{自荐信页面}
	自荐信页在\emp{通用配置.tex}中没有需要特殊设置的选项,但有一点要说明:自荐信的署名和署名日期,用的是\emp{通用配置.tex}的作者名和发布日期。若您需要更改,则自荐信页中将其删除,写上实际姓名和日期即可。
	\begin{latexcode}
	\begin{flushright}
		自荐人\quad hazukie
		2025年8月
	\end{flushright}
	\end{latexcode}
	
	考虑手动替换容易导致更新不及时,\underline{在配置文件新增了自定义选项。}以下是具体步骤:
	\begin{ul}
	\item 打开\emp{通用配置.tex},找到以下选项:
	\begin{latexcode}
	\def\theCoverLetterCutomNeed{true} %是否开启自定义填写
	\def\theCoverLetterAuthor{hazukie}  %自荐信的作者姓名
	%自荐信的日期:自行填写|\today
	\def\theCoverLetterDate{2025年8月8日}  
	\end{latexcode}
	
	\item 点按编译
	\end{ul}

	\subsubsection{简历页}
	这里是简历模板中重点之重点,考虑到每个人对于内容排列次序和样式的要求均有所区别,故这里仅介绍:通用组成次序、可用命令。当\emp{可用命令}不能满足定制需求时,可前往第 \ref{sec:extension} 节作进一步了解。
	
	对简历编写只做概略性说明,只是方便后续排版解释的展开。
	
	通常,简历版块有:基本信息、教育经历、实习经历、项目经历、专业技能、荣誉证书、个人评价。
	其中,以上版块还可以根据个人实际情况,进行合并与拆分。比如:\emp{个人评价}可以合并到教育经历;若校园有任职团支书等,可增加\emp{校园经历}。每一个版块又可以细分若干,以列表罗列,着重强调者又以加粗表示。可知,$\mbox{版块}=\mbox{版块标题}+\mbox{版块内容(若干)}$。
	
	简历的基本信息版块,在模板中是固定的,其信息需要在\emp{通用配置.tex}中进行修改。若想要更换,您需要利用 LaTeX 中的表格等来重新设计,这里就不提供现成样式以供选择了。通用配置中选项的相关说明,请阅读最后一章节。
	
	\noindent\textbf{排版解释}
	
	版块:由标题和内容区组成。使用命令:\textbackslash spect\{标题\}。
	\begin{latexcode}
	\spect{教育经历} %一个版块
	%这里是版块内容区
	
	\spect{实习经历} %另一个版块
	%这里是版块内容区
	\end{latexcode}
	\fpic{spect.png}
	
	版块内容:可以是 自然段落、列表、罗列性子项。
	\begin{ul}
		\item 自然段落
		
		使用:在 \textbackslash spect\{标题\} 的下一行开始写,隔一行表示另一个自然段落开始。但在简历中默认不缩进,若要调整缩进,请使用:\textbackslash pind 表示缩进2个字符。
		\begin{latexcode}
	\spect{教育经历} %一个版块
	这里是版块内容区。
	
	另一段落开始...
	\pind 另一段缩进
		\end{latexcode}
		\fpic{pitem.png}
	
		\item 列表
		
		由一个概述性条目+描述性片段组成[可选]。
		
		使用命令:
		
		\begin{latexcode}
		\litem{概述}{时间/其他说明}
		\end{latexcode}
		\fpic{litem-2.png}
		
		\begin{latexcode}
 \litem{概述}{时间/其他说明}
 \descript{
   \indent 积极参与学生会的各项活动,领导其他干事一起参与各类活动策划
 }
		\end{latexcode}
		\fpic{litem-2.png}

		\item 罗列性子项
		
		使用命令:
		
		\textbackslash begin\{pitem/pblock/plist\}
		
		\qquad \textbackslash item 内容...
		
		\textbackslash end\{pitem/pblock/plist\}
		
		\begin{latexcode}
	\subsection{pitem罗列性子项}
	\begin{pitem}
		\item 内容1...
		\item 内容2...
		\item 内容3...
	\end{pitem}
	
	\subsection{pblock罗列性子项}
	\begin{pblock}
		\item 内容1...
		\item 内容2...
		\item 内容3...
	\end{pblock}	

	\subsection{plist罗列性子项}
	\begin{plist}
		\item 内容1...
		\item 内容2...
		\item 内容3...
	\end{plist}
	
		\end{latexcode}
		\fpic{litems.png}
		通过图片的红线,可以清晰地看到三个罗列性子项的缩进是不同的,pitem、pblock、plist 分别是缩进2个字符、1个字符、0个字符。你可以将这三种形式进行组合以得到合理排版。
	
	\end{ul}
	
	\newpage
	版块布局:默认为单栏,但也可以是双栏、三栏等。多栏排版可以使用 \textbackslash multicols\{栏数\} 命令得到,同时将前面命令进行嵌套组合。
	
	\quad 使用命令:
	
	\qquad\textbackslash begin\{multicols\}\{n\}
	
	\qquad 内容1...
	
	\qquad 内容2
	
	\qquad \textbackslash end\{multicols\}
	\begin{latexcode}
	\begin{multicols}{2}
    LaTeX 已经成为国际上数学、物理、计算机等科技领域专业排版的实际标准,
    其他领域(化学、生物、工程等)也有大量用户。
		
    本书内容取材广泛,涵盖了正文组织、自动化工具、数学公式、
    图表制作、幻灯片演示、错误处理等方面。
	\end{multicols}
	\end{latexcode}
	\fpic{nsides.png}
	
	细节调整:留白、字体、段落间距、数值单位
	\begin{ul}
		\item 留白:可以使用以下命令,以调整上下左右的空白宽度。
		\begin{table}[!ht]
			\centering
			\zihao{5}
			\begin{tabular}{l|l|l|l}
				\hline
				\textbackslash quad & \textbackslash qquad & \textbackslash ; & \textbackslash \textbackslash\\ 
				空一格 & 空两格 & 空半格 & 换行,但没有留白\\ \hline
				\textbackslash hspace\{n pt/cm\} & \textbackslash vspace\{n pt/cm\}&
				\textbackslash hspace*\{n pt/cm\} & \textbackslash vspace*\{n pt/cm\}\\
				横向留白& 垂直留白& 强制版& 强制版\\
				\hline
			\end{tabular}
		\end{table}
		\item 字体
		
		字体大小:中文可以使用 \textbackslash zihao\{字号\},正数是\emp{n号字体},负数是\emp{小n号字体}。英文也是可以用\textbackslash zihao 的,但也有特定的指令可以用 \textbackslash large,\textbackslash small等。
		当你只想对某些字生效时,请用括号将内容括起来:\{\textbackslash zihao\{n\} 内容 \}。 
		
		字体选择:一般有 \textbackslash kaishu,\textbackslash heiti,\textbackslash songti,分别是 楷书、黑体、宋体。使用时可以和\textbackslash zihao 结合起来,如:\{\textbackslash kaishu\textbackslash zihao\{2\} 标题\},就表示将二号字体的楷书应用到\emp{标题}两个汉字上。
		
		自定义字体:请看第\ref{sec:extension}节进一步了解。
		
		\item 段落
		直接空一行,只表示新的段落开始。当想强制指定段落间距时可以用 \textbackslash setlength \textbackslash parskip\{n pt/cm\}。部分生效,则用括号\{\}括起来。
		换行则用 \textbackslash\textbackslash,当然也可以在换行后指定上下距离: \textbackslash\textbackslash[距离 pt/cm]。
		
		\item 数值和单位说明:
		\textbf{注意,数值可以正数,可以是负数的!}
		
		同时,单位有:pt、em、cm、mm 等。特别说明:em 是一个\emp{m}的宽度。\emp{\%} 是注释符号,\%后面的内容表示这部分不是正文,仅用来解释说明提示。百分号,请用转义符号\emp{\textbackslash \%}
	\end{ul}
	\subsubsection{附件索引页}
	由于默认不开启\emp{编译时不运行脚本}选项,所以您将会看到此页面为空白。需要开启,请前往\emp{通用配置.tex}找到选项并修改为:
	\begin{latexcode}
	\def\theAttachIndexScriptEmbededNeed{true}
	\end{latexcode}
	
	然后重新编译即可看到索引内容。
	
	手动编译方法:
	\begin{ul}
		\item 打开终端
		\item 汇总指令: make collect/rcollect/ncollect
		\item 编译指令: make
		\item 查看文件
	\end{ul}
	\subsubsection{附件页}
	此页面主要插入奖状、技能证书、毕业证等文件,支持 JPG、PNG、PDF等格式。
	
	这里,将详细介绍插入竖版文件、横版文件、整页文件(如PDF)的指令,分别是 \textbackslash vpic(竖版)、\textbackslash hpic(横版)、\textbackslash ppic(整页插入)	。
	\textbackslash vpic、\textbackslash hpic 指令需要填写三个参数,分别为 文件路径、名称、授予日期,\underline{后两个参数为可选,若没有内容}则保持空括号\emp{\{\}}。\textbackslash ppic 则为四个参数,分别是 文件缩放系数、文件路径、名称、授予日期。指令格式为:
	\begin{latexcode}
		\vpic/hpic{文件路径}{名称}{日期}
		
		\ppic{文件缩放系数}{文件路径}{名称}{日期}	
	\end{latexcode}

	出于美观考虑,竖版文件每页排两张,横版文件则排三张。
	
	竖版文件排版格式(建议):
	\begin{latexcode}
	\begin{figure}[!h]
		\centering
		\vpic{./imgs/vertical-sample.jpg}{大学英语四级证书}{2021.12}
		\vspace*{2em}
		\vpic{./imgs/vertical-sample.jpg}{大学英语六级证书}{2022.12}
	\end{figure}
	\end{latexcode}
	\fpic{vpic-sample.png}

	横版文件排版格式(建议):
	\begin{latexcode}
	\begin{figure}[!h]
		\centering
		\hpic{./imgs/horizontal-sample.jpg}{原神大学位证书}{2025.07}
		\vspace*{1em}
		\hpic{./imgs/horizontal-sample2.jpg}{原神大学毕业证书}{2025.07}
		\vspace*{1em}
		\hpic{./imgs/horizontal-sample3.png}{原神大学计算机辅修证书}{2025.04}
	\end{figure}
	\end{latexcode}
	\fpic{hpic-sample.png}
	
	整页文件排版格式(建议):
	\begin{latexcode}
	\clearpage %新开一页
	\ppic{}{./imgs/vertical-sample.pdf}{本科成绩单}{2025.04}
	\end{latexcode}
	\newpage
	\fpic{single-page-sample.pdf}

	\newpage
	\section{自定义拓展}\label{sec:extension}
	欢迎来到自定义拓展章节,在这里你将会懂得:
	\begin{ul}
	\item 如何添加额外包(Package)
	\item 自定义命令和常量(Command、Constant)
	\item 自定义字体
	\item 添加书签、超链接
	\end{ul}
	
	同时,在章节最后有进一步阅读的书籍清单以供参考。
	
	\subsection{添加额外包}
	每个人的需求各式各样,文中提供的命令必定不能覆盖。所以,您在制作简历时,会频繁地添加第三方包(Third Package),以实现您的意图。
	添加方式也很简单,只需要在\emp{通用配置.tex} 的末尾增加:
	\begin{latexcode}
	\usepackage{包名}
	\end{latexcode}
	然后,返回正文去使用相关命令即可。
	
	欲了解相关包的使用和指令参数,您可以:
	\begin{ul}
		\item 在终端上输入:
		\begin{bashcode}
		texdoc 包名
		\end{bashcode}
		
		\item 网络上键入并回车 \href{https://ctan.org/}{https://ctan.org/},然后在搜索框键入\emp{包名},即可找到相关包的帮助页面。
		\item 咨询 AI(ChatGPT、DeepSeek)或者在浏览器(推荐 Edge、Chrome)搜索 \emp{latex \textcolor{red}{包名}使用},在查看相关信息时须留意是否过时。
		\item \LaTeX 论坛,如:\href{https://tex.stackexchange.com/}{LaTeX Stack Exchange} 等。
	\end{ul}
	
	\subsection{自定义命令和常量}
	自定义命令名称使用大小写混合方式书写,可以选择传递参数或者不传递参数,\emp{参数n}在内容中用\emp{\#n}调用。
	在\emp{通用配置.tex} 的末尾增加:
	\begin{latexcode}
	\newcommand{\指令名称1}{内容}
	\newcommand{\指令名称2}[参数个数(假设为2)]{
		... #1 %参数1
		... #2 %参数2
	}
	\end{latexcode}
	然后在正文中,
	\begin{latexcode}
	\指令名称1 
	\指令名称2{参数1}{参数2}
	\end{latexcode}

	自定义常量,和自定义命令类似:
	在\emp{通用配置.tex} 的末尾增加:
	\begin{latexcode}
		\def\常量名称1{内容}
		\def\常量名称2#1#2{
			这是参数 #1, %参数1
			还有这个 #2。 %参数2
		}
	\end{latexcode}
	然后在正文中,
	\begin{latexcode}
		\常量名称1 
		\常量名称2 参数1 {参数2}
	\end{latexcode}
	注意,定义的名称不能和现有指令冲突,否则可能会发生不可预料的错误导致编译失败。
	
	\subsection{自定义字体}
	自定义字体命令格式:
	\begin{latexcode}
	\newCJKfontfamily{\名称}{字体文件名称}[Path=字体文件路径]
	\end{latexcode}

	您需要将下载的字体文件移动到当前文件夹的\emp{fonts}文件夹中。
	这里假设该字体文件名称为\emp{custom-heiti.ttf}

	在\emp{通用配置.tex}末尾中添加:
	\begin{latexcode}
	\newCJKfontfamily{\cheiti}{custom-heiti.ttf}[Path=fonts/]
	\end{latexcode}
	然后返回正文中,使用 \textbackslash cheiti 即可。
	
	\subsection{书签、超链接}
	书签,指在阅读 PDF 文件时,侧边栏有书签显示(相当于目录),点击也能跳转到相关内容处。值得注意的是 \textbackslash spect、\textbackslash zTitle 已经添加有书签标记,\underline{请不要重复添加!}
	
	在您想要添加书签的正文\underline{内容上方},添加:
	\begin{latexcode}
		\tagBookmark{书签名}
	\end{latexcode}
	然后点按编译(\textbf{需要编译2次}),用阅读器打开 PDF 文件即可看到新的书签。
	
	超链接指令:\textbackslash href\{网址\}\{名称\}。在正文中使用此指令即可。
	
	\subsection{书籍阅读推荐清单}
	\begin{ul}
		\item \href{https://texdoc.org/serve/lshort-zh-cn.pdf/0}{一份(不太)简短的 \LaTeXe 介绍}
		\item \href{https://github.com/huangxg/lnotes.git}{关于\LaTeXe 的系列笔记}
		\item \href{https://zilutian.github.io/latex-tutorial-chinese/}{latex-tutorial-chinese}
		\item \href{https://github.com/wklchris/Note-by-LaTeX}{简单粗暴 LATEX}
		\item \href{https://math.ecnu.edu.cn/~jypan/Latex/index.html}{\LaTeX 科技排版}
		\item \href{https://lrita.github.io/images/wiki/latex入门-简版-刘海洋.pdf}{\LaTeX 入门}
	\end{ul}

	\newpage
	\section{通用配置清单}
	% 配置文件的表格
	\begin{table}[!h]
		\centering

		\begin{tabularx}{\textwidth}{|l|l|>{\raggedright\arraybackslash}X|}
			\hline
			\textbf{配置名}                    & \textbf{可填项}                                        & \textbf{说明}                      \\
			\hline
			\texttt{\textbackslash theCoverPageNeed}        & \texttt{true/false}                               & 是否需要封面页                   \\
			\hline
			\texttt{\textbackslash CoverPageBgNeed}        & \texttt{true/false}                               & 是否显示封面背景色               \\
			\hline
			\texttt{\textbackslash theCoverPageBgPath}     & 路径名                & 封面背景图片路径                 \\
			\hline
			\texttt{\textbackslash theCoverPageCustomNeed} & \texttt{true/false}                               & 是否需要自定义封面               \\
			\hline
			\texttt{\textbackslash theCoverPageCustomPath} & \texttt{路径名}& 自定义封面的文件路径             \\
			\hline
			\texttt{\textbackslash theCoverPageCustomScale} & \texttt{0-1}& 自定义封面的缩放系数             \\
			\hline
			\texttt{\textbackslash theBgNeed}              & \texttt{true/false}                               & 是否需要背景色                   \\
			\hline
			\texttt{\textbackslash theBgPath}              & \texttt{路径名}& 背景色路径                       \\
			\hline
			\texttt{\textbackslash theCoverLetterNeed}     & \texttt{true/false}                               & 是否需要自荐信                   \\
			\hline
			\texttt{\textbackslash theCoverLetterCutomNeed} & \texttt{true/false}                               & 是否开启自定义填写               \\
			\hline
			\texttt{\textbackslash theCoverLetterAuthor}   & \texttt{姓名}& 自荐信的作者姓名                 \\
			\hline
			\texttt{\textbackslash theCoverLetterDate}     & \texttt{日期}& 自荐信的日期                     \\
			\hline
			\texttt{\textbackslash theAttachmentIndexPageNeed} & \texttt{true/false}                            & 是否需要附件索引页               \\
			\hline
			\texttt{\textbackslash theAttachIndexScriptEmbededNeed} & \texttt{true/false}                      & 是否在编译时调用汇总脚本         \\
			\hline
			\texttt{\textbackslash theAttachCollectWay}    & \texttt{[n|r| ]collect}               & 无序/逆序/顺序                 \\
			\hline
			\texttt{\textbackslash theAttachmentPageNeed}  & \texttt{true/false}                               & 是否需要附加页                   \\
			\hline
			\texttt{\textbackslash theUniversityName}      & \texttt{大学名称}&\\
			\hline
			\texttt{\textbackslash theCollegeName}         & \texttt{学院名称}&\\
			\hline
			\texttt{\textbackslash theSession}             & \texttt{xxxx届}&\\
			\hline
			\texttt{\textbackslash theMajor}               & \texttt{专业名称}&\\
			\hline
			\texttt{\textbackslash theAuthorName}          & \texttt{作者姓名}&\\
			\hline
			\texttt{\textbackslash theAge}                 & \texttt{年龄}&\\
			\hline
			\texttt{\textbackslash theGender}              & \texttt{性别}&\\
			\hline
			\texttt{\textbackslash theHeight}              & \texttt{身高}&\\
			\hline
			\texttt{\textbackslash theHometown}            & \texttt{籍贯}&\\
			\hline
			\texttt{\textbackslash thePolitics}            & \texttt{政治面貌}&\\
			\hline
			\texttt{\textbackslash theStudyMode}           & \texttt{学制+学历}&\\
			\hline
			\texttt{\textbackslash thePhoneNumber}         & \texttt{联系方式}&\\
			\hline
			\texttt{\textbackslash theEmail}               & \texttt{电子邮件地址}&\\
			\hline
			\texttt{\textbackslash theReleaseDate}         & \texttt{发布日期}& \\
			\hline
			\texttt{\textbackslash theAvatarPath}          & \texttt{证件照路径名}&                \\
			\hline
		\end{tabularx}
	\end{table}
	
	
	\newpage
	\section{已添加包清单}
	\begin{table}[!ht]
		\centering
		\renewcommand{\arraystretch}{1.2}
		\begin{tabularx}{\textwidth}{
				|l|>{\centering\arraybackslash}X|}
			\hline
			\textbf{包名}& \textbf{用途}\\\hline
			xeCJK & 支持中文排版 \\\hline
			graphicx & 图像处理 \\ \hline
			pdfpages & 插入 PDF 页面 \\ \hline
			tikz & 绘图工具 \\ \hline
			titlesec & 自定义标题格式 \\\hline 
			enumitem & 自定义列表样式 \\ \hline
			caption & 自定义图表标题 \\ \hline
			etoolbox & 条件控制 \\ \hline
			wallpaper & 壁纸背景设置 \\ \hline
			multicol & 多栏页面排版 \\ \hline
			newclude & 加载不分页的合并宏 \\ \hline
			hyperref & 超链接支持 \\ \hline
		\end{tabularx}
	\end{table}

	\newpage
	\section{可用命令清单}
	% \newcommand 命令的表格
	\begin{table}[ht]
		\centering
		\renewcommand{\arraystretch}{1.2}
		\begin{tabularx}{\textwidth}{|l|>{\raggedright\arraybackslash}X|>{\raggedright\arraybackslash}X|}
			\hline
			\textbf{命令} & \textbf{用途} & \textbf{参数说明} \\
			\hline
			\texttt{\textbackslash sectionline} & 绘制一条水平线 & 无 \\
			\hline
			\texttt{\textbackslash rTitle\{\#1\}} & 定义一个居中的大标题 & \texttt{\#1} - 标题内容 \\
			\hline
			\texttt{\textbackslash zTitle\{\#1\}} & 定义一个居中的次级大标题 & \texttt{\#1} - 标题内容 \\
			\hline
			\texttt{\textbackslash litem\{\#1\}\{\#2\}} & 定义左对齐的列表项 & \texttt{\#1} - 列表前的条目,\texttt{\#2} - 对应的内容 \\
			\hline
			\texttt{\textbackslash fsong\{\#1\}} & 使用仿宋体显示文本 & \texttt{\#1} - 显示的文本 \\
			\hline
			\texttt{\textbackslash spect\{\#1\}} & 版块标题命令,带有下划线 & \texttt{\#1} - 标题内容 \\
			\hline
			\texttt{\textbackslash pitem} & 左空2格的列表 & 无 \\
			\hline
			\texttt{\textbackslash pblock} & 左空1格的列表 & 无 \\
			\hline
			\texttt{\textbackslash plist} & 左无空格的列表 & 无 \\
			\hline
			\texttt{\textbackslash descript\{\#1\}} & 使用特定字号描述文本 & \texttt{\#1} - 要描述的内容 \\
			\hline
			\texttt{\textbackslash pind} & 左缩进2格 & 无 \\
			\hline
			\texttt{\textbackslash vpic\{\#1\}\{\#2\}\{\#3\}} & 设置竖向插图命令 & \texttt{\#1} - 图片路径,\texttt{\#2} - 图片标题,\texttt{\#3} - 可选日期 \\
			\hline
			\texttt{\textbackslash hpic\{\#1\}\{\#2\}\{\#3\}} & 设置为横向插图 & \texttt{\#1} - 图片路径,\texttt{\#2} - 图片标题,\texttt{\#3} - 可选日期 \\
			\hline
			\texttt{\textbackslash ppic\{\#1\}\{\#2\}\{\#3\}\{\#4\}} & 插入一页完整的 PDF & \texttt{\#1} - 插入选项,\texttt{\#2} - 文件路径,\texttt{\#3} - 文件标识,\texttt{\#4} - 可选日期 \\
			\hline
			\texttt{\textbackslash addCoverBg} & 添加封面背景 & 无 \\
			\hline
			\texttt{\textbackslash addBg} & 添加文档背景 & 无 \\
			\hline
			\texttt{\textbackslash tagBookmark\{\#1\}} & 手动添加书签 & \texttt{\#1} - 书签名称 \\
			\hline
		\end{tabularx}
	\end{table}

	\newpage
	\section{可用自定义字体清单}	
	% 自定义字体的表格
	% 自定义字体的表格
	\begin{table}[ht]
		\centering
		\renewcommand{\arraystretch}{1.2}
		\begin{tabularx}{\textwidth}{|c|>{\raggedright\arraybackslash}X|}
			\hline
			\textbf{字体名称}       & \textbf{命令}                        \\
			\hline
			Times New Roman        & \textbackslash timesnr \\
			\hline
			黑体                  & \textbackslash zhei\\
			\hline
			思源黑体              & \textbackslash hansc \\
			\hline
			仿宋体                & \textbackslash thefsong\\
			\hline
		\end{tabularx}
	\end{table}	
	
	\newpage
	\section{致谢}
	在此,我衷心感谢所有为《简易简历模板》提供帮助与支持的人。首先,我要感谢我的家人和朋友们,感谢他们在我学习和研究期间给予的鼓励和支持,使我能够坚持不懈并最终完成此项目。
	
	我还想特别感谢那些分享了他们的知识和经验的前辈和网友们,他们的宝贵建议和专业意见提升了教程的质量,帮助我更好地理解LaTeX的使用。感谢网络社区的成员们,特别是在Stack Exchange和GitHub等平台上无私分享经验的开发者们,你们的热情和共享精神让我在面对困难时充满信心。
	
	最后,对所有曾经使用和反馈此教程的读者表示诚挚的感谢,正是你们的参与和建议才让这个教程更加完善。我希望本教程能为更多人提供帮助,让他们在制作简历的过程中更加得心应手。
	
	\begin{flushright}
		叶月绘梨依\quad 落笔
		
		2025年8月9日
	\end{flushright} 
\end{document}